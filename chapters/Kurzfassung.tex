
\kurzfassung

%% deutsch
\paragraph*{}
%In der Kurzfassung soll in kurzer und prägnanter Weise der wesentliche Inhalt der Arbeit beschrieben werden. Dazu zählen vor allem eine kurze Aufgabenbeschreibung, der Lösungsansatz sowie die wesentlichen Ergebnisse der Arbeit. Ein häufiger Fehler für die Kurzfassung ist, dass lediglich die Aufgabenbeschreibung (d.h. das Problem) in Kurzform vorgelegt wird. Die Kurzfassung soll aber die gesamte Arbeit widerspiegeln. Deshalb sind vor allem die erzielten Ergebnisse darzustellen. Die Kurzfassung soll etwa eine halbe bis ganze DIN-A4-Seite umfassen.
%\newline
%Hinweis: Schreiben Sie die Kurzfassung am Ende der Arbeit, denn eventuell ist Ihnen beim Schreiben erst vollends klar geworden, was das Wesentliche der Arbeit ist bzw. welche Schwerpunkte Sie bei der Arbeit gesetzt haben. Andernfalls laufen Sie Gefahr, dass die Kurzfassung nicht zum Rest der Arbeit passt.

Die vorliegende Arbeit nimmt sich zum Ziel, die Funktionsprinzipien und Anwendungen der Pfadplanung zu erläutern. Dazu wird anfänglich die Vorgehensweise bei der Pfadplanung beschrieben und die wichtigsten Grundlagen erklärt. Die Algorithmen werden am Beispiel der Pfadplanung im diskreten Zustandsraum eingeführt und auf die Unterschiede eingegangen. Der kontinuierliche Zustandsraum wird hingegen bei den Anwendungen erläutert und durch Anwendungsbeispiele veranschaulicht.
Es wird dadurch ein Grundverständis erreicht, um im folgenden selbst entscheiden zu können, in welches Teilgebiet der Pfadplanung sich die Leserin weiterführend einarbeiten sollte, um Projekte umzusetzen.

%% englisch
%\paragraph*{}
%The same in english.
