\kurzfassung

%% deutsch
\paragraph*{}
%In der Kurzfassung soll in kurzer und pr"agnanter Weise der wesentliche Inhalt der Arbeit beschrieben werden. Dazu z"ahlen vor allem eine kurze Aufgabenbeschreibung, der L�sungsansatz sowie die wesentlichen Ergebnisse der Arbeit. Ein h"aufiger Fehler f"ur die Kurzfassung ist, dass lediglich die Aufgabenbeschreibung (d.h. das Problem) in Kurzform vorgelegt wird. Die Kurzfassung soll aber die gesamte Arbeit widerspiegeln. Deshalb sind vor allem die erzielten Ergebnisse darzustellen. Die Kurzfassung soll etwa eine halbe bis ganze DIN-A4-Seite umfassen.
%\newline
%Hinweis: Schreiben Sie die Kurzfassung am Ende der Arbeit, denn eventuell ist Ihnen beim Schreiben erst vollends klar geworden, was das Wesentliche der Arbeit ist bzw. welche Schwerpunkte Sie bei der Arbeit gesetzt haben. Andernfalls laufen Sie Gefahr, dass die Kurzfassung nicht zum Rest der Arbeit passt.

Die vorliegende Arbeit nimmt sich zum Ziel die Funktionsprinzipien und Anwendungen der Pfadplanung zu erl"autern. Dazu wird die anf"anglich die Vorgehensweise bei der Pfadplanung beschrieben und die wichtigsten Grundlagen erkl"art. Die R"aumliche Darstellung der Umgebung und verschiedene Typen von Graphen werden unter anderem am Beispiel von Potentialfeld und Navigationsnetzen erl"autert, um ein weiterf"uhrendes Verst"andnis zu schaffen auf welches im folgenden zur"uckgegriffen wird. Es wird darauf geachtet sowohl Methoden in einem diskreten sowie in einem kontinuierlichen Raum darzustellen. 
Weiterhin wird auf Algorithmen zur Pfadplanung eingegangen, hier wird ein Einblick in die Diskrete Pfadplanung, Bewegungsplanung, Planen mit Unsicherheit und Planen mit differentialen Einschr"ankungen gegeben, mit dem Ziel "uber die algorithmischen Herausforderungen und Eigenheiten der Teilgebiete zu informieren. 
Themengebiet zu gro� ist um genauer auf einzelne Themen einzugehen wird auch immer wieder auf weiterf"uhrende Literatur verwiesen.
Abschlie�en werden die Einsatzgebiete der Algorithmen mit konkreten Anwendungen konkretisiert und anwendungsbezogene Herausforderungen werden erl"autert. Da das  Es wird dadurch ein Grundverst"andis erreicht, um im folgenden selbst Entscheiden zu k"onnen, in welches Teilgebiet der Pfadplanung sich die Leserin weitergehend einarbeiten sollte um Projekte umzusetzen. 

%% englisch
%\paragraph*{}
%The same in english.
