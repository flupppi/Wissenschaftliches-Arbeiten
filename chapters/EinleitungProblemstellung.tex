\chapter{Einleitung und Problemstellung}

Die Pfadplanung in unserem Leben wird immer relevanter. Ob in der realen Welt, oder in virtuellen Welten, steigt die Notwendigkeit/Verwendung von Pfadplanungsalgorithmen immer weiter. Sowohl in der Unterhaltungsindustrie wie z.B. Videospielen, als auch in der Wirtschaft zum digitalen Planen von Maschinen und Fabriken kommen Pfadplanungsalgorithmen vor. Aber auch in unseren Alltag halten sie in Form von Navigationssystemen und Autopiloten Einzug. Die Pfadplanung stellt einen Roboter vor verschiedenste Probleme, darunter fallen das Verst"andnis der Umwelt, Pfadfindung, Bewegung, Kollisionsvermeidung und weitere. Um diese Aufgaben auszuf"uhren muss der Roboter eine Vorgehensweise f"ur jeder haben, also ein Algorithmus. Dazu muss eine f"ur den Roboter verst"andliche Darstellung der Umwelt existieren, die f�r diese Aufgaben genutzt werden kann. Um den korrekten Algorithmus f"ur die Anwendung w"ahlen zu k"onnen, muss herausgearbeitet werden wie diese sich unterscheiden. 
