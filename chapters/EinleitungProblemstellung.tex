
\chapter{Einleitung und Problemstellung}

Die Pfadplanung in unserem Leben wird immer relevanter. Ob in der realen Welt, oder in virtuellen Welten, steigt die Notwendigkeit/Verwendung von Pfadplanungsalgorithmen immer weiter. Sowohl in der Unterhaltungsindustrie wie z.B. Videospielen, als auch in der Wirtschaft zum digitalen Planen von Maschinen und Fabriken kommen Pfadplanungsalgorithmen vor. Aber auch in unseren Alltag halten sie in Form von Navigationssystemen und Autopiloten Einzug. Die Pfadplanung stellt einen Roboter vor verschiedenste Probleme. Darunter fallen das Verständnis der Umwelt, Pfadfindung, Bewegung, Kollisionsvermeidung und Weitere. Um diese Aufgaben auszuführen muss der Roboter eine Vorgehensweise für jede haben, also einen Algorithmus. Dazu muss eine für den Roboter verständliche Darstellung der Umwelt existieren, die für diese Aufgaben genutzt werden kann. Um den korrekten Algorithmus für die Anwendung wählen zu können, muss herausgearbeitet werden wie diese sich unterscheiden. 
