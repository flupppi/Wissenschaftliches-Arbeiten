
\chapter{Einleitung und Problemstellung}
 
Pfadplanung ist ein essenzieller Bestandteil von intelligenten Computersystemen. 
Sowohl in der Unterhaltungsindustrie wie z.B. Videospielen, als auch in der Wirtschaft zum digitalen Planen von Maschinen und Fabriken kommen Pfadplanungsalgorithmen vor. 
In den Alltag halten sie in Form von Navigationssystemen und Autopiloten Einzug.
Ein immer wiederkehrender Begriff bei der Pfadplanung ist der \textit{Roboter}. 
Hier wird die Definition von Steven Lavalle aus \cite[S. 4]{Lav06} verwendet.
Er definiert ihn als den Nutzer eines Plans, mit dem der Roboter Entscheidungen trifft. 
Gleich bedeutende Begriffe sind auch \textit{agent} oder \textit{player}. 
Die Pfadplanung stellt einen Roboter vor verschiedenste Problemstellungen.
Darunter fallen das Verständnis der Umwelt, Pfadfindung, Bewegung, Kollisionsvermeidung und weitere. 
Algorithmen geben dem Roboter eine Handlungsanweisung zur Überwindung jener.
Dazu muss eine, für ihn verständliche, Darstellung der Umwelt existieren.
Um den korrekten Algorithmus für die Anwendung wählen zu können, muss herausgearbeitet werden, welche Problemstellungen die Anwendungen an ihn stellen. 

