\chapter{Glossar}

\abbreviation{Turing Maschine}		{Eine Turingmaschine modelliert die Arbeitsweise eines Computers auf besonders einfache und mathematisch gut zu analysierende Weise. Sie ist benannt nach dem Mathematiker Alan Turing, der sie 1936 einführte\cite{wiki:01}.  }
\abbreviation{Church-Turing These}{\grqq Alles was intuitiv berechenbar ist, d.h. alles, was von einem Menschen berechnet werden kann, das kann auch von einer Turingmaschine berechnet werden. Ebenso ist alles, was eine andere Maschine berechnen kann, auch von einer Turingmaschine berechenbar\cite{Heitmann:16}.\grqq{}} 
\abbreviation{Turing vollständig}			{\grqq Ein Berechnungsmodell, mit dem alle Funktionen
	der Klasse WHILE berechnet werden können, heißt
	Turing-vollständig \cite[Definition 5.4]{Schmitz:19}.\grqq{}}
\abbreviation{Zustandsraum}			{In der theoretischen Informatik ist ein Zustandsraum eine Beschreibung von diskreten Zuständen, um sie als einfaches Modell von Maschinen zu verwenden (z. B. Endliche Automaten)\cite{wiki:02}}
\abbreviation{abzählbar unendlich}	{Eine Menge $M$ heißt abzählbar unendlich, falls $M \approx \mathbb{N}$. Eine Menge heißt abzählbar, falls sie endlich oder abzählbar unendlich ist. Andernfalls heißt die Menge	überabzählbar\cite[Definititon 5.8]{Schmitz:19}.}
\abbreviation{Bewegungseinschränkung}{		     .   (engl. differential constraints)}
\abbreviation{FP} {feasible planning (Dt. durchführbares Planen)}
\abbreviation{First-In First-Out} {Das am längsten wartende Element zuerst}
\abbreviation{Last-In First-Out} {Das neuste Element zuerst}
