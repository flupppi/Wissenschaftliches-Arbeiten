% !TeX encoding = ISO-8859-1
\chapter{Zusammenfassung und Ausblick}

F�r Pfadplanung sind die Arbeitsraum und die Konfigurationsraum der Roboter wichtig, weil mit ihnen seine Position in der reale Welt beschreibt werden kann. Um den Raum darzustellen gibt es mehrere Optionen: Potentialfeld, Roadmaps und Zelldekomposition. Und um diese zu Traversieren ist ein Graph notwendig, weil sie f�r Pfadfindung genutzt wird.

Desweiteren wurde auf die Funktionsweise der Algorithmen eingeganen, hier sind vorallem die Abgrenzung zwischen Maschine und Umwelt erw�hnenswert, durch die R�presentation eines Pfadplangungsalgorithmus als ein Planer, der einen Plan produziert verdeutlicht wurde. Die drei M�glichkeiten einen Plan zu verwenden wurden erkl�rt und dabei auch der Verbesserungsprozess aus der Robotik behandelt. Die Algorithmen wurden in einen kontinuierlichen und einen diskreten Zustandsraum unterteilt. Anhand von diesem wurden M�glichkeiten zur Unterscheidung erl�utert und dabei Dijkstra und A-Stern erl�utert. 

Es wurden verschiedene Anwendungsbeispiele f�r Pfadplanungalgorithmen sowohl im diskreten, als auch im kontinuierlichen Zustandsraum und jeder der Unterteilungen letzterer gebracht  um diese zu veranschaulichen.
