\chapter{Zusammenfassung und Ausblick}

F"ur Pfadplanung sind der Arbeitsraum und der Konfigurationsraum der Roboter wichtig, weil mit ihnen seine Position in der realen Welt beschrieben werden kann. Um den Raum darzustellen, gibt es mehrere Optionen: Potentialfeld, Roadmaps und Zelldekomposition. Und um diese zu traversieren, ist ein Graph notwendig, weil dieser f�r Pfadfindung genutzt wird.

Desweiteren wurde auf die Funktionsweise der Algorithmen eingegangen. Hier ist vor allem die Abgrenzung zwischen Maschine und Umwelt erw"ahnenswert, durch welche die Repr"asentation eines Pfadplanungsalgorithmus als ein Planer, der einen Plan produziert verdeutlicht, wurde. Die drei M"oglichkeiten, einen Plan zu verwenden, wurden erkl"art und dabei auch der Verbesserungsprozess aus der Robotik behandelt. Die Algorithmen wurden in einen kontinuierlichen und einen diskreten Zustandsraum unterteilt. Anhand von diesen wurden M"oglichkeiten zur Unterscheidung erl"autert und dabei Dijkstra und A-Stern erl"autert. 

Es wurden verschiedene Anwendungsbeispiele f�r Pfadplanungsalgorithmen sowohl im diskreten, als auch im kontinuierlichen Zustandsraum und dessen Unterteilungen gebracht, um diese zu veranschaulichen.
