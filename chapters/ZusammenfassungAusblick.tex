
\chapter{Zusammenfassung und Ausblick}
In dieser Arbeit wurden wichtige Definitionen der Pfadplanung vorgestellt. Es wird mit dem Konfigurationsraum begonnen, der die Welt für Roboter beschreibt. Die Darstellung der Welt mit Potentialfeldern, Roadmaps und Zelldekompositionen wurde erläutert. Graphen werden zur Fortbewegung in dieser Welt verwendet. Es wurden Algorithmen zur Pfadplanung definiert und eine Klassifizierung dieser vorgenommen. Dabei wurden Unterschiede zwischen diskreten und kontinuierlichen Zustandsräumen herausgearbeitet. Beim FP wurde die Sortierung der Warteschlange als primäres Unterscheidungsmerkmal herausgearbeitet. Es wurden Beispiele in verschiedenen Anwendungsbereichen der Pfadplanung aufgezeigt und diese in die vorgenommene Klassifizierung eingeordnet.\\
Aus unserer Ausarbeitung wird erkennbar, dass an die Algorithmen je nach Anwendungsgebiet unterschiedliche Anforderungen gestellt werden. Deshalb ist es sinnvoll, sich speziell in das für sich selbst relevante Teilgebiet bzw. die relevante Anwendung einzuarbeiten. Hierfür kann sich näher mit den von uns genannten Quellen befasst werden.
