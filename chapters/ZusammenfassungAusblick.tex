
\chapter{Zusammenfassung und Ausblick}

Für Pfadplanung sind die Arbeitsraum und die Konfigurationsraum der Roboter wichtig, weil mit ihnen seine Position in der reale Welt beschreibt werden kann. Um den Raum darzustellen gibt es mehrere Optionen: Potentialfeld, Roadmaps und Zelldekomposition. Und um diese zu Traversieren ist ein Graph notwendig, weil sie für Pfadfindung genutzt wird.

Desweiteren wurde auf die Funktionsweise der Algorithmen eingeganen, hier sind vorallem die Abgrenzung zwischen Maschine und Umwelt erwähnenswert, durch die Räpresentation eines Pfadplangungsalgorithmus als ein Planer, der einen Plan produziert verdeutlicht wurde. Die drei Möglichkeiten einen Plan zu verwenden wurden erklärt und dabei auch der Verbesserungsprozess aus der Robotik behandelt. Die Algorithmen wurden in einen kontinuierlichen und einen diskreten Zustandsraum unterteilt. Anhand von diesem wurden Möglichkeiten zur Unterscheidung erläutert und dabei Dijkstra und A-Stern erläutert. 

Es wurden verschiedene Anwendungsbeispiele für Pfadplanungsalgorithmen sowohl im diskreten, als auch im kontinuierlichen Zustandsraum und dessen Unterteilungen gebracht, um diese zu veranschaulichen.
