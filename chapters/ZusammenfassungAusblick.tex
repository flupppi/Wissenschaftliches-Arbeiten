
\chapter{Zusammenfassung und Ausblick}

Für Pfadplanung sind der Arbeitsraum und der Konfigurationsraum der Roboter bedeutend, weil mit ihnen ihre Position in der realen Welt beschrieben werden kann. Um den Raum darzustellen, gibt es mehrere Optionen: Potentialfelder, Roadmaps und Zelldekompositionen. Um diese zu traversieren ist jeweils ein Graph notwendig, wie er in der Pfadfindung genutzt wird.
%TODO What?? MUSS verbssert werden.
Des weiteren wurde auf die Funktionsweise der Algorithmen eingegangen. Hier ist vor allem die Abgrenzung zwischen Maschine und Umwelt erwähnenswert, durch welche die Repräsentation eines Pfadplanungsalgorithmus als ein Planer, der einen Plan produziert, verdeutlicht wurde. Die drei Möglichkeiten einen Plan zu verwenden wurden erklärt und dabei auch der Verbesserungsprozess aus der Robotik behandelt. Es wurde dargestellt, dass kontinuierliche und diskrete Zustandsräume jeweils unterschiedlicher Algorithmen bedürfen. Mit Hilfe dieser Unterscheidung wurden Dijkstra und A-Stern erläutert. 

Zur Veranschaulichung wurden verschiedene Anwendungsbeispiele für Pfadplanungsalgorithmen sowohl im diskreten, als auch im kontinuierlichen Zustandsraum gebracht.
%TODO Ausblick
Um konkreter in einzelne Themengebiete der Pfadplanung einzusteigen, empfehlen wir, sich eingehend mit den hier genannten Quellen auseinander zu setzen.