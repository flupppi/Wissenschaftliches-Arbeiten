\chapter{Anwendungen von Algorithmen}
%Grundeinleitung
Nachdem die Grundlagen der Funktionsprinzipien der Algorithmen zu Pfadplanung nun erl"autert  wurden, wird nun auf m"ogliche Anwendungen dieser eingegangen.

\section{Anwendungen mit diskretem Zustandsraum}
Die Klassifizierung des Planens im  diskreten Zustandsraum ist z.B. f"ur das Rubik's Cube R"atsel zutreffend (siehe Abb. \ref{Abb. 5.1}), bei dem es ebenso einen endlichen Zustandsraum und einen endlichen Aktionsrahmen gibt. Der Zustandsraum ist hierbei die Summe aller Zust"ande, die der W"urfel annehmen kann, also jegliche m"ogliche Farbverteilung. Der Aktionsrahmen ist die Menge aller Richtungen, in die man jedes Element drehen kann. Der bestm"ogliche Pfad ist in diesem Fall die geringste Anzahl an Drehungen des W"urfels in der richtigen Reihenfolge, um ans Ziel bzw. zum Zielzustand zu gelangen, welcher ist, dass jede Seite des W"urfels nur noch Elemente einer Farbe hat.\cite[~S. 20]{Lav06}\\
\begin{figure}
	\centering
	\includegraphics[width=0.4\linewidth]{images/img229}
	\caption{in Anlehnung Abb. 1.1 von \cite[~S. 5]{Lav06}: Rubik's Cube.}
	\label{Abb. 5.1}
\end{figure}

Die n"achste Anwendung ist die Bewegungen eines Roboters oder einer K"unstlichen Intelligenz auf einem 2D Netz. Dies geschieht in Strategie-Computerspielen, in denen eine Figur von ihrem aktuellen Standpunkt zu einem Ort gelangen muss, der ihr zugewiesen wird. Hier kommt der Pfadplanungsalgorithmus zum Einsatz.\cite{cui2011based}%genauer drauf neue Quelle eingehen.\\
%s27ff
\section{Anwendungen zur Planung mit kontinuierlichem Zustandsraum}
Wie in Kapitel \ref{Kapitel 4.2} schon angeschnitten klassifiziert man bei den Algorithmen zur Planung mit kontinuierlichem Zustandsraum zwischen drei verschiedenen Rubriken, die unterschiedliche Anwendungsbereiche haben, auf die im Folgenden einzeln eingegangen wird.
\subsection{Anwendungen zur Planung mit allen Umgebungsinformationen vorhanden}
Ein gutes Beispiel f"ur Anwendungen zur Planung, bei der alle Umgebungsinformationen vorhanden sind, ist das bereits in \ref{Kapitel 4.2} angesprochene Piano Mover's Problem.\\
Eine praktische Anwendung ist das digitale Planen von Fabrik-Robotern, die ein Auto zusammensetzen sollen. In Abb. \ref{Abb. 5.2} ist das digitale Planen des automatischen An- und Abmontierens eines Scheibenwischermotors an ein Auto mittels eines CAD Modells dargestellt.
\begin{figure}
	\centering
	\includegraphics[width=0.7\linewidth]{images/img231}
	\caption{Abb. 1.3 von \cite[~S. 7]{Lav06}: Eine Auto-Montierungsaufgabe, die beinhaltet, dass ein Scheibenwischermotor an ein Auto angebracht oder entfernt werden soll.}
	\label{Abb. 5.2}
\end{figure}

Eine Software muss entscheiden, ob und wie der Scheibenwischermotor An- und abmontiert werden kann, oder nicht. Das fr"uher zeit- und kostenintensive Entwickeln des Designs wird nun mit Software vereinfacht, indem CAD Modelle manipuliert werden. Auch hier hat man einen "uberabz"ahlbar unendlichen Zustandsraum und alle Umgebungsinformationen sind bekannt. \cite[~S. 6 ff]{Lav06}
%part 2 QUELLE ANGEBEN
\subsection{Anwendungen zur Planung mit Unsicherheit}
Bei der Planung mit Unsicherheit (engl. planning under uncertanty), auch decision-theoretic planning genannt, interferieren Ungewissheiten im Allgemeinen mit 2 Aspekten des Planens. Diese beiden Aspekte sind Vorhersehbarkeit und Wahrnehmung.\\
Die Vorhersehbarkeit ist in dem Sinne durch die Unsicherheiten beeintr"achtigt, dass nicht bekannt ist, was passieren wird, gewisse Aktionen ausgef"uhrt werden und die Wahrnehmung ist durch die Unsicherheiten beeintr"achtigt, da der aktuelle Status, bzw. die aktuelle Position nicht bekannt ist. Denn den Status erh"alt man aus den Ausgangsbedingungen, den Sensoren und den Informationen "uber die vorangegangenen Aktionen.\\
Ein Beispiel f"ur eine solche Anwendung ist der Tesla Autopilot oder Autonomes Fahren im Allgemeinen.%Quellen suchen
 Hier bewegt sich das Auto nicht "uber ein festes 2D Netz, sondern durch die unvorhersehbare Realit"at. Der Bordcomputer des Autos nimmt durch seine Sensoren und eventuelle Au"senkameras seine Umgebung war und versucht durch diese und GPS-Daten seinen Status bzw. seine aktuelle Position herauszufinden. Da sich jedoch die Umgebung stets ver"andert und der Verhalten der anderen Verkehrsteilnehmer nicht vorhersagbar ist, kommt die Unsicherheit ins Spiel.
Dies l"asst sich auf die Navigation von allen mobilen Robotern "ubertragen.
Da der aktuelle Status nie gewiss sein kann, wird mit allen verf"ugbaren Informationen versucht, den Status so genau wie m"oglich zu bestimmen.
%(s.25) Part 3
\subsection{Anwendungen zur Planung mit Bewegungseinschr"ankungen}
Bei Anwendungen zur Planung mit Bewegungseinschr"ankungen (engl. planning under differential constraints) muss zus"atzlich aber noch beachtet werden, dass der schnellste Pfad, der errechnet wird, oft durch Bewegungseinschr"ankungen nicht umgesetzt werden kann. In der Robotik entstehen diese Bewegungseinschr"ankungen h"aufig durch die Kinematik und Dynamik der Roboter selbst.\\
Auch hier kann gut Auto als Beispiel herhalten. Ein Auto kann nicht seitw"arts fahren. Wenn es also seitlich neben einer Parkl"ucke steht und man den Einparkassistenten das Auto einparken lassen will, w"are der k"urzeste Pfad seitw"arts zu fahren. Da die nicht m"oglich ist, muss das bei der Pfadplanung ber"ucksichtigt werden(siehe Abb. \ref{Abb. 5.3}).
\begin{figure}
	\centering
	\includegraphics[width=0.5\linewidth]{images/img239}
	\caption{in Anlehnung an Abb. 1.11 von \cite[~S. 15]{Lav06}}
	\label{Abb. 5.3}
\end{figure}

%s26 Part IV